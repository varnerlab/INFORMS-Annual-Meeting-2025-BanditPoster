\documentclass[aspectratio=169]{beamer}

%Information to be included in the title page:
% \title{Sample title}
% \author{Anonymous}
% \institute{Overleaf}
% \date{2021}

% \usepackage{xcolor}
% \usepackage[dvipsnames]{xcolor}
% \setbeamercolor{background canvas}{bg=black!50}
% \setbeamercolor{\texttt}{fg=white}\usebeamercolor*{\texttt}

\usepackage{algpseudocode}
\usepackage{algorithm}

\hypersetup{
    colorlinks=true,
    linkcolor=blue,
    filecolor=magenta,      
    urlcolor=cyan,
    pdftitle={Overleaf Example},
    pdfpagemode=FullScreen,
}
\DeclareMathOperator*{\argmin}{arg\,min}
\DeclareMathOperator*{\argmax}{arg\,max}

\setbeamertemplate{itemize items}[circle]
% \setbeamercolor{block title}{use=structure,fg=white,bg=structure.fg!75!black}

% 1- Block title (background and text)
% \setbeamercolor{block title}{bg=cyan, fg=white}
% % 2- Block body (background)
% \setbeamercolor{block body}{bg=cyan!10}

\begin{document}

\begin{frame}
\textbf{Problem Setup:} Portfolio $\mathcal{P}$ of risky assets with prices $p_i$ and availability $a_i \in \{0,1\}$. Allocate budget $B$ to maximize portfolio utility.
\end{frame}

\begin{frame}
\textbf{Utility Function:} Cobb-Douglas utility measures investor satisfaction:
\begin{align*}
U(n_1, \dots, n_P) = \kappa(\gamma) \prod_{i \in \mathcal{P}} n_i^{\gamma_i}
\end{align*}
where $\gamma_i$ is the preference for asset $i$ (output of a preference model), $\kappa(\gamma)$ scales the function, and $n_i$ is the number of shares (what we compute).
\end{frame}

\begin{frame}
\textbf{Investor preference model:} The $\gamma_{i}$ preference coefficients can reflect market conditions, sentiment, and other asset-specific information through a feature vector $\mathbf{x}_{i}\in\mathbb{R}^{m}$:
\begin{align*}
\gamma_{i} & = \sigma\left(\mathbf{x}^{\top}_{i}\theta_{i}\right)\quad\forall{i}\in\mathcal{P}
\end{align*}
where $\sigma:\mathbb{R}\rightarrow\mathbb{R}$ is an activation function $\sigma_{\theta}(x)\in[-1,1]$,
and $\mathbf{\theta}_{i}\in\mathbb{R}^{p}$ ($p=m+1$) denotes the feature weights (and bias), learned from data or set based on subjective beliefs.
\end{frame}

\begin{frame}
Allocate investment budget $B$ across portfolio assets, where $p_i$ is the price of asset $i$.
The optimal portfolio is the solution of the utility maximization problem:
\begin{align*}
    \underset{n_{1},\dots,n_{P}}{\text{maximize}} &\quad \kappa(\gamma)\prod_{i\in\mathcal{P}}n_{i}^{\gamma_{i}} \\
    \text{subject to}&\quad B =  \sum_{i\in\mathcal{P}}n_{i}\;{p}_{i}\\
    \epsilon\;{a_{i}}&\leq n_{i} \leq{a_{i}}\left(\frac{B}{p_{i}}\right)\quad{\forall{i}\in\mathcal{P}}\\
    a_{i} &\in\{0,1\}\quad{\forall{i}\in\mathcal{P}}\\
    \epsilon &\in\mathbb{R}_{+} \quad\text{(hyperparameter)}
\end{align*}
\end{frame}

\begin{frame}
The share optimization problem has an \textbf{analytical solution}. 
Let $S = \{i \mid a_i = 1\}$ be the available assets; $S_+ = \{i \mid \gamma_i > 0\}$ the preferred ones, $S_- = \{i \mid \gamma_i < 0\}$ the non-preferred.
The optimal shares $n_i^*$ are:
\begin{align*}
n_{i}^{\star} & = \begin{cases}
\left(\frac{\gamma_{i}}{\sum_{j\in{S}_{+}}\gamma_{j}}\right)\;\frac{B - \epsilon\sum_{k\in{S}_{-}}p_{k}}{p_{i}} & \forall{i}\in{S}_{+}\\
\epsilon & \forall{i}\in{S}_{-}
\end{cases}\quad\blacksquare    
\end{align*}
Here, $n_i^*$ is the optimal number of shares for asset $i$, balancing preferences and budget constraints.
\end{frame}

\end{document}