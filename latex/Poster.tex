\documentclass[aspectratio=169]{beamer}

%Information to be included in the title page:
% \title{Sample title}
% \author{Anonymous}
% \institute{Overleaf}
% \date{2021}

% \usepackage{xcolor}
% \usepackage[dvipsnames]{xcolor}
% \setbeamercolor{background canvas}{bg=black!50}
% \setbeamercolor{\texttt}{fg=white}\usebeamercolor*{\texttt}

\usepackage{algpseudocode}
\usepackage{algorithm}

\hypersetup{
    colorlinks=true,
    linkcolor=blue,
    filecolor=magenta,      
    urlcolor=cyan,
    pdftitle={Overleaf Example},
    pdfpagemode=FullScreen,
}
\DeclareMathOperator*{\argmin}{arg\,min}
\DeclareMathOperator*{\argmax}{arg\,max}

\setbeamertemplate{itemize items}[circle]
% \setbeamercolor{block title}{use=structure,fg=white,bg=structure.fg!75!black}

% 1- Block title (background and text)
% \setbeamercolor{block title}{bg=cyan, fg=white}
% % 2- Block body (background)
% \setbeamercolor{block body}{bg=cyan!10}

\begin{document}

\begin{frame}
There exists a collection of risky assets in the portfolio $\mathcal{P}$.
An oracle provides the current price $p_{i}\in\mathbb{R}_{+}$ for each asset $i\in\mathcal{P}$, and 
a binary action vector $a\in\{0,1\}^{|\mathcal{P}|}$ indicating whether each asset is available for investment ($a_{i}=1$) or not ($a_{i}=0$).
The goal of the investment agent is to allocate a fixed budget $B$ across these assets to \textbf{maximize the utility} of the portfolio. 
\end{frame}

\begin{frame}
\textbf{Utility Function:} Cobb-Douglas utility measures investor satisfaction:
\begin{align*}
U(n_1, \dots, n_P) = \kappa(\gamma) \prod_{i \in \mathcal{P}} n_i^{\gamma_i}
\end{align*}
where $\gamma_i$ is the preference for asset $i$ (output of a preference model), $\kappa(\gamma)$ scales the function, and $n_i$ is the number of shares (what we compute).
\end{frame}

\end{document}